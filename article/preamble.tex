
\usepackage{graphicx}      % include this line if your document contains figures
\usepackage{grffile}       % filenames
\usepackage{booktabs}       % filenames
\usepackage{amsmath}
\usepackage{mathtools}
\usepackage{amssymb}
% \usepackage[showframe, pass]{geometry}
% \usepackage{amsfonts}
\usepackage{hyperref}
\usepackage{tikz}
\usepackage[math]{blindtext}
\usepackage{tikzscale}
\usepackage{bm}
\usepackage{ifthen}
\usepackage[ruled,noend]{algorithm2e}
% \usepackage[justification=centering]{caption}
% \usepackage{subcaption}
\SetKwRepeat{Do}{do}{while}
\usepackage[author={Accacio}]{pdfcomment}

\newif\ifdebug%


\usepackage[acronym]{glossaries}%

\newcommand{\acrSing}[3]{\newacronym{#1}{#2}{#3}
  \expandafter\newcommand\csname #1\endcsname{\gls{#1}}}

\newcommand{\acrPl}[5]{
  \newacronym[plural=#4,firstplural=#5 (#4)]{#1}{#2}{#3}
  \expandafter\newcommand\csname #1\endcsname{\gls{#1}}
  \expandafter\newcommand\csname #4\endcsname{\glspl{#1}}
}
\newcommand{\acr}[5][4=,5=]{
  % \ifthenelse{\equal{#5}{}}
  % {
    \acrSing{#1}{#2}{#3}
  % }
  % {
  %   \acrPl{#1}{#2}{#3}{#4}{#5}
  % }
}

% \newcommand{\symbl}[3]{\newglossaryentry{#1}{name ={#2},	description ={#3}}
%   \expandafter\newcommand\csname #1\endcsname{\gls{#1}}
% }




\usepackage{color}
\newcommand{\no}[1]{}

\newcommand{\comments}{
  \usepackage{geometry}
  \newgeometry{
    textwidth=\columnwidth,
    % left=60mm,
    hoffset=-1.5in,
    bottom=0.41in,
    top=0.41in
    % marginparwidth=413pt
  }
  \onecolumn
}
\newcommand{\draft}{\debugtrue}
\newcommand{\final}{\debugfalse}
\newcommand{\question}[1]{\ifdebug%
  {\pdfcomment[color=red,opacity=0.4,subject=Should I?]{#1}} \fi}
\newcommand{\todo}[2][FORGOT TO DO SOMETHING]{\ifdebug%
  {\color{red}#2}\else \PackageError{}{#1}{#2}#2\fi}%
\newcommand\doing[2][FORGOT TO DO SOMETHING]{\ifdebug%
  {\color{blue}#2}\else \PackageError{}{#1}{#2}#2\fi}%
\newcommand\warning[1]{\ifdebug%
  {\color{red}#1}\fi}
\newcommand\note[1]{\ifdebug%
  {\pdfcomment[color=orange,opacity=0.4,subject=note]{#1}} \fi}


%===============================================================================
\newtheorem{theorem}{theorem}[section]
\newtheorem{problem}{Problem}%[numberby]
\newtheorem{example}{Example}%[numberby]
\newtheorem{remark}{Remark}%[numberby]
\newtheorem{assumption}{Assumption}%[numberby]

\newcommand{\ensemblenombre}[1]{\mathbb{#1}}
\newcommand{\eq}[2]{\mbox{$#1=#2$}}
\newcommand{\NN}{\mathbb{N}}
\newcommand{\ZZ}{\mathbb{Z}}
\newcommand{\QQ}{\mathbb{Q}}
\newcommand{\RR}{\mathbb{R}}
\newcommand{\CC}{\mathbb{C}}
\newcommand{\Np}{N_{\text{p}}}
\newcommand{\T}{^{\mathrm{T}}}
\newcommand{\1}{\mathbf{1}}
\newcommand{\0}{\mathbf{0}}
\newcommand{\abs}[1]{\left\lvert#1\right\rvert}
\newcommand{\norm}[1]{\|#1\|}
\newcommand{\Varepsilon}{\mathcal{E}}
\newcommand{\diff}{\mathop{}\mathopen{}\mathrm{d}}
\newcommand{\set}[1]{\mathcal{#1}}
\newcommand{\p}{^{(p)}}
\newcommand{\pplusone}{^{(p+1)}}
\renewcommand{\vec}[1]{\bm{#1}}
\newcommand{\enstq}[2]{\{#1\mathrel{}\mid\mathrel{}#2\}}

\graphicspath{{../img/}}
